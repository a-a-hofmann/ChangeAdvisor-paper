%User feedback plays a paramount role in the development and maintenance of mobile applications. 
In order to hit the market as quick as possible and to win market share, the software development industry has embraced a paradigm of short and iterative development cycles. Especially in the world of mobile applications, where the advent of mobile app stores has made it easier than ever to manage updates. With the app stores, came also the possibility of users to voice their opinion regarding software in the form of short texts and numerical scores. These reviews are free text and may contain information relevant for developers~\cite{inukollu2014factors}, albeit informally and unstructured. The information contained ranges from problem reports~\cite{Pagano:2013:RE}, to feedback regarding features~\cite{Guzman:2014:RE}, requests for enhancements~\cite{iacob2013retrieving} and new features~\cite{Pagano:2013:RE,galvis2013analysis}, and comparisons with competing apps.
Thus app stores provide provide people a way to give their opinion in a quick and simple manner, but it is also a powerful crowd feedback mechanism, which enables developers to discover and fix potential bugs as soon as possible, as well as redirecting the project direction to fulfill market requirements.

Thus, as a natural consequence, research started to come up with ways to capitalize on this wealth of information. Harman \textit{et al.}~\cite{Harman:MSR:2012} introduced the concept of \textit{App Store Mining} as a form of \textit{Mining Software Repositories}.

There are difficulties in exploiting this information, however. The various app stores, usually provide only limited support for sorting and aggregating this data to derive requirements~\cite{abelein2013does}. Considering the sheer amount of reviews popular apps receive, it is sometimes not feasible at all to read them all, even less to distinguish which ones are relevant for maintenance tasks.

Thus researchers practitioners have come up with new approaches to distinguish relevant reviews~\cite{iacob2013retrieving,panichella2015can,oh2013facilitating}. Many of these approaches however only classify reviews, without considering semantics. Additionally there exists no way of correlating user feedback to the actual code components that need to be changed.